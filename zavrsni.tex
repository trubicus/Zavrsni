\documentclass[times, utf8, zavrsni]{fer}
\usepackage{booktabs}

\begin{document}

% TODO: Navedite broj rada.
\thesisnumber{5417}

% TODO: Navedite naslov rada.
\title{Implementacija sustava za nadziranje i upravljanje bežičnim razvojnim modulima ESP8266}

% TODO: Navedite vaše ime i prezime.
\author{Ivan Trubić}

\maketitle

% Ispis stranice s napomenom o umetanju izvornika rada. Uklonite naredbu \izvornik ako želite izbaciti tu stranicu.
\izvornik

% Dodavanje zahvale ili prazne stranice. Ako ne želite dodati zahvalu, naredbu ostavite radi prazne stranice.
\zahvala{}

\tableofcontents

\chapter{Uvod}
Uvod rada. Nakon uvoda dolaze poglavlja u kojima se obrađuje tema.

\chapter{ESP8266}
Opis modula i navedene razne prednosti i mane.

\chapter{Usporedba drugih razvojnih modula koji koriste WiFi komunikacijski protokol sa ESP8266 modulom}
Usporedba raznih arduina i slicnih mikrokontrolera koji se mogu spojiti na WiFi.

\chapter{Programska podrska za pracenje komunikacije vise modula ESP8266 na WiFi mrezi}

\chapter{Zaključak}
Zaključak.

\bibliography{literatura}
\bibliographystyle{fer}

\begin{sazetak}
Sažetak na hrvatskom jeziku.

\kljucnerijeci{Ključne riječi, odvojene zarezima.}
\end{sazetak}

% TODO: Navedite naslov na engleskom jeziku.
\engtitle{Management Framework for ESP8266 WiFi Development Modules}
\begin{abstract}
Abstract.

\keywords{Keywords.}
\end{abstract}

\end{document}
